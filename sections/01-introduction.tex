\section{Introduction}

\subsection*{Grupo de estudos de ROS 2}
\begin{frame}{Grupo de Estudos de ROS 2}
    \begin{columns}
        \begin{column}{0.6\textwidth}
            \begin{itemize}
                \item Trainees 2022.1
                \item Membros do Raspode e Mapvision
                \item Será aberto para todos que quiserem participar(mesmo que não faça parte do IEEE)
            \end{itemize}
        \end{column}

        \begin{column}{0.4\textwidth}
            \begin{figure}
                \centering
                \includegraphics[height = 0.6\textheight]{img/ros2-foxy.png}
                \caption{Logo ROS 2 - Foxy}
            \end{figure}
        \end{column}

    \end{columns}
\end{frame}

\subsection*{Processo Trainee}

\begin{frame}{Processo Trainee}
    \begin{columns}
        \centering
        \begin{column}{0.4\textwidth}
            \textbf{1° Etapa} \\
            \small
            Deverá ser feito um post sobre os projetos da capacitação de Arduino.

        \end{column}
        \begin{column}{0.5\textwidth}
            \textbf{2° Etapa} \\
            \small
            O projeto de conclusão deve ser entrega juntamente com um artigo.\\
            Com a retomada das atividades presenciais as equipes devem agendar um horário na semana para trabalhar com o projeto no LabMaker.
        \end{column}
    \end{columns}
\end{frame}

\subsection*{SIINTEC VIII}

\begin{frame}{SIINTEC VIII}
    \begin{columns}
        \centering
        \begin{column}{\textwidth}
            \textbf{Submissões} \\
            \small
            1- A ROBOTIC PLATFORM FOR ASSISTANCE IN THE MEDICAL TRIAGE PROCESS\\
            2- JABUTI PROJECT: MAZE SOLVER MICROMOUSE ROBOT

        \end{column}
        \begin{column}{0.001\textwidth}
\end{column}
    \end{columns}
\end{frame}

\section{Eventos}

\subsection*{MicroRAS}
\begin{frame}{MicroRAS}
    \begin{columns}
        \begin{column}{0.4\textwidth}
            \begin{figure}
                \centering
                \includegraphics[height = 0.35\textheight]{img/siintec.jpeg}
                \caption{Logo SIINTEC VIII.}
            \end{figure}
        \end{column}

        \begin{column}{0.4\textwidth}
            \begin{figure}
                \centering
                \includegraphics[height = 0.25\textheight]{img/ford.png}
                \caption{Logo da Ford.}
            \end{figure}
        \end{column}

    \end{columns}
\end{frame}

\begin{frame}{MicroRAS}
    \begin{columns}
        
        \begin{column}{0.6\textwidth}
            Recrutamento para:\\
            \begin{itemize}
                \item Organização do evento;
                \item Manufatura dos robôs;
                \item Produção do material didático.
            \end{itemize}
        \end{column}

        \begin{column}{0.4\textwidth}
            \begin{figure}
                \centering
                \includegraphics[height = 0.35\textheight]{img/micro-ras.png}
                \caption{Logo da Competição MicroRAS.}
            \end{figure}
        \end{column}

    \end{columns}
\end{frame}

\section{Projetos}
\begin{frame}{Requisitos}
    \begin{columns}
        
        \begin{column}{0.6\textwidth}
            O líder do projeto deve:
            \begin{itemize}
                \item Cronograma de atividades;
                \item O que cada voluntário fará;
                \item Dias e horários das reuniões(presenciais).
            \end{itemize}
        \end{column}

        \begin{column}{0.4\textwidth}

        \end{column}

    \end{columns}
\end{frame}
\subsection*{Baymax}
\subsection*{Petieee}
\subsection*{Raspode}
\subsection*{Jabuti}
\subsection*{Mapvision}

\section{Comitês}
\subsection*{Eventos}
\subsection*{Produção de Conteúdo}
